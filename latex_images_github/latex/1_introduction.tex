\chapter{前言}
\renewcommand{\baselinestretch}{10.0} %設定行距
\pagenumbering{arabic} %設定頁號阿拉伯數字
\setcounter{page}{1}  %設定頁數
\fontsize{14pt}{2.5pt}\sectionef
\section{設計架構}
此次 pj3 專題目標有建立場景中的計時器、球員外型及移動優化、添加球員擊球和翻車再起能、進球後收集並隨機投下新的一顆球、建立以機械式轉盤傳動計分系統。由於目標繁多,需要組員間分工負責,在每個禮拜的協同中逐步完成 pj3 專題。\\

\begin{figure}[hbt!]
\begin{center}
\includegraphics[width=15cm]{35}
\caption{\Large 設計目標圖}\label{fig.35}
\end{center}
\end{figure}
\section{規則說明}\emph{}
類似於足球遊戲,一開始時球會置於場中央,遊戲開始後雙方即可以鍵盤操控機器人,透過防守敵方以及與隊友間的傳球推球至已方的球門得分。\\
遊戲規則如下:
\begin{enumerate}
\item 球觸碰到球門感測器即算得分。
\item 在十分鐘的比賽時間內,獲得最多分數的隊伍即獲勝。
\item 任一方進球得分後,隨機在場內投下新的球,雙方接續進行比賽。
\end{enumerate}

\renewcommand{\baselinestretch}{0.5} %設定行距